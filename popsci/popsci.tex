
% USE PDFLatex!

\documentclass{popsci}

\usepackage[utf8]{inputenc}
\usepackage[swedish, english]{babel}

\usepackage{fancyhdr}
\usepackage{titling}
\usepackage{color}
\usepackage{colortbl}

\usepackage{lmodern}



\presentationsdag{2017-08-25}

\examensarbete{Safe test selection for Modelica using static analysis}
%To create a title in two rows, leave examensarbete blank and fill in examensarbeteTwoRows.
\examensarbeteTwoRows{}{}

\author{Erik Hedblom, Kasper Rundqvist}
\supervisor {Niklas Fors (LTH), Johan Ylikiiskilä (Modelon)}
\examiner{Görel Hedin (LTH)}

\title{50\% snabbare testning för Modelica}


\begin{document}

\theabstract{Applikations-specifika processorer är allt mer vanligt för få ut rätt prestanda med så lite resurser som möjligt. Detta arbete har en parametrisk modell för att kunna testa hur mycket resurser som behövs för en specifik applikation.}


{\noindent Det finns mycket tid och resurser att spara om det går att minska testningen för mjukvara. Under utveckling av mjukvara så testas mjukvaran för att kontrollera att den fungerar som det är tänkt. För att vara säker på att något test som visar på ett fel i mjukvaran inte missas, är det lättaste alternativet att välja att köra alla test som finns vid testning. Men om det endast har gjorts en liten ändring så kan det vara så att många test körs helt i onödan, om de inte testar någon del av mjukvaran som har påverkats av ändringen. Vi har utvecklat en teknik för att analysera mjukvara och välja bort tester som är onödiga, för att kunna minska tiden testningen tar utan att försämra kvaliteten på testningen.
}

\end{document}
