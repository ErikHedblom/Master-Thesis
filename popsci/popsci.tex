
% USE PDFLatex!

\documentclass{popsci}

\usepackage[utf8]{inputenc}
\usepackage[swedish, english]{babel}

\usepackage{fancyhdr}
\usepackage{titling}
\usepackage{color}
\usepackage{colortbl}

\usepackage{lmodern}



\presentationsdag{2017-08-25}

\examensarbete{Safe test selection for Modelica using static analysis}
%To create a title in two rows, leave examensarbete blank and fill in examensarbeteTwoRows.
\examensarbeteTwoRows{}{}

\author{Erik Hedblom, Kasper Rundquist}
\supervisor {Niklas Fors (LTH), Johan Ylikiiskilä (Modelon)}
\examiner{Görel Hedin (LTH)}

\title{50\% snabbare testning för Modelica}


\begin{document}

\theabstract{Många företag lägger mycket tid och resurser på att testa mjukvara. Med tekniken vi har utvecklat för att välja ut tester har vi lyckats minska tiden för testning med 56 \% utan att försämmra kvaliteten på testningen.}


{\noindent Det finns mycket tid och resurser att spara om det går att minska testningen för mjukvara. När mjukvara utvecklas används tester för att kontrollera att programmet fungerar som det är tänkt. Under utvecklingens gång så måste gamla tester köras om för att kontrollera att nya funktioner inte har ändrat beteendet för gamla funktioner. Samtidigt skapas nya tester för de nya funktionerna. Detta betyder att det kan bli många, väldigt många tester som kontinuerligt måste köras. Vår teknik kan minska antal tester som behöver köras när en ny funktion läggs till och spara mer än 50\% av testtiden. 

För att vara säker på att ny funktionalitet inte förstör gamla funktioner så är det enklast att köra om alla tester. Detta är dock som att byta alla säkringar i ett hus för att en säkring har löst ut. Många tester kommer då att köras helt i onödan om inte den nya funktionaliteten påverkar hela programmet. Vår teknik analyserar istället programmet och tar reda på vilka delar som kan tänkas ha påverkats. Sedan räcker det med att bara köra tester för de berörda delarna av programmet. Vår teknik garanterar att de tester som inte körs inte heller kan misslyckas.

Tekniken vi har utvecklat är specifik för programmeringsspråket Modelica men liknande teknik har tidigare använts för vanligare programmeringsspråk som till exempel Java. Modelica är ett modelleringsspråk som används för simuleringar inom många olika områden, bl.a. simulering av elektriska komponenter, motorer och fordon. 

För att ta reda på hur mycket av tiden som går åt till testning som kan sparas så har vi tittat på de historiska ändringarna i ett Modelica-projekt. Vi har jämfört hur mycket tid som hade sparats om vår teknik för att välja ut tester hade använts istället för att köra alla tester. Vi har kommit fram till att tiden för testning hade minskats med 56 \% av tiden det tar att köra alla test för detta projekt. Detta är en stor tidsbesparing eftersom det tog två och en halv timme att köra alla tester.

När ett program utvecklas så införs många ändringar varje dag. Att köra alla tester för varje ändring är därför oftast inte möjligt. För varje gång testerna körs kan det ha hunnit komma in många nya ändringar. Men vår teknik kan vi minska tiden för varje testomgång och ge snabbare återkoppling till utvecklarna. Om ett test misslyckas så kan det även vara lättare att hitta vilken ändring som som är orsaken eftersom färre ändringar är med i varje testomgång. 
}

\end{document}
